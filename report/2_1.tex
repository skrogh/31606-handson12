\subsection{Get rid of it}
Now that we have our \emph{STFT} and \emph{ISTFT} functions, it is time to put
them to use.

We load a wave file, signal $s$, with a sencence about the cocktail effect (the
abilty to idetify a single speaker in an ocean of noise).

Performing an \emph{STFT} on the signal with the following settings:
\begin{itemize}
  \item Window length: $256$
  \item Window type: $\sqrt{\mbox{hann}}$
  \item Zero pad to length: $256 \cdot 4 = 1024$
  \item Overlap of $50\%$
\end{itemize}

We can plot the spectrogram of this signal $S$, as seen in figure
\ref{fig:2-1-signal}
\stdfignoscale{2-1-signal}{Spectrogram of the signal
\emph{cocktail.wav}}{fig:2-1-signal}

We also generate some gaussian white noise $nosie$ and scale this so:
$\sqrt{ \mbox{mean}( noise^2 ) } = \sqrt{ \mbox{mean}( s^2 ) } \cdot (-10
\mbox{dB})$

We name the \emph{STFT} of this signal $NOISE$.
This noise has the spectrogram seen in figure \ref{fig:2-1-noise}
\stdfignoscale{2-1-noise}{Spectrogram of the noise. Note that
even though this is white noises, the spectrum is not
compleatly flat, but looks more like a tv-screen with no signal}{fig:2-1-noise}

Adding these two signals in the time domain we get the signal $s_{+noise} = s
+ noise$, with the \emph{STFT} $S_{+noise}$. This signal has the spectrum seen
in figure \ref{fig:2-1-signal-noise}
\stdfignoscale{2-1-signal-noise}{Spectrogram of the signal + noise.}{fig:2-1-signal-noise}

\paragraph{Denoising}
To denoise the signal, we estimate the noise floor and cut out anything below
some value above this floor.

The noise floor is estimated at:
\begin{equation*}
\widehat{NOISE} = \mbox{mean}(NOISE)
\end{equation*}

We then define $3$ gating levels: $LVL_1 = \widehat{NOISE}$, $LVL_2 =
2\widehat{NOISE}$, $LVL_3 = 3\widehat{NOISE}$

And use this to create the denoised \emph{STFT}s:
\begin{gather*}
\widehat{SIGNAL}_1 = \begin{cases}
      				S_{+noise}, & \text{if}\ \abs{S_{+noise}} > LVL_1 \\
      				0, & \text{otherwise}
    			 \end{cases}
\\
\widehat{SIGNAL}_2 = \begin{cases}
      				S_{+noise}, & \text{if}\ \abs{S_{+noise}} > LVL_2 \\
      				0, & \text{otherwise}
    			 \end{cases}
\\
\widehat{SIGNAL}_3 = \begin{cases}
      				S_{+noise}, & \text{if}\ \abs{S_{+noise}} > LVL_3 \\
      				0, & \text{otherwise}
    			 \end{cases}
\end{gather*}

Spectrograms of these denoised signals can be seen in figure
\ref{fig:2-1-est1}, \ref{fig:2-1-est2}, and \ref{fig:2-1-est3}.
Note how the level if noise is reduced, but part of the speach signal is also
removed. Especially the high frequencies (because they have low magnitude) and
the consonant (because they are mostly noise and thus the energy is spread wide
in the frequency, leading to a low magintude) are compromised.

\stdfignoscale{2-1-est1}{Spectrogram of the denoised signal for a
gating level of $\widehat{NOISE}$}{fig:2-1-est1}
\stdfignoscale{2-1-est2}{Spectrogram of the denoised signal for a
gating level of $2\widehat{NOISE}$}{fig:2-1-est2}
\stdfignoscale{2-1-est2}{Spectrogram of the denoised signal for a
gating level of $3\widehat{NOISE}$}{fig:2-1-est3}


